\documentclass[a4paper, 11pt]{article}
\usepackage{times}
\usepackage[left=2cm,top=3cm,text={17cm, 24cm}]{geometry}

\usepackage[utf8]{inputenc}
\usepackage[IL2]{fontenc}
\usepackage[czech]{babel}

\usepackage{caption}
\usepackage{amsmath}
\usepackage{multirow}
\usepackage[linesnumbered, ruled, czech]{algorithm2e}
\usepackage{picture}

\usepackage{graphics}

\begin{document}

\begin{center}
\thispagestyle{empty}
\Huge{\textsc{Vysoké učení tehcnické v Brně}\\
\Huge{Fakulta informačních technologií}}\\\vspace{\stretch{0.382}}
\LARGE{Typografie a publikování – 3. projekt}\\ 
\Huge{Tabulky a obrázky} \\\vspace{\stretch{0.618}}
\end{center}
\LARGE{20. března 2019 \hfill Daniel Bílý}


\normalsize
\pagebreak
\setcounter{page}{1}
\section{Úvodní strana}
Název práce umístěte do zlatého řezu a nezapomeňte uvést dnešní datum a vaše jméno a příjmení.

\section{Tabulky}
Pro sázení tabulek můžeme použít buď prostředí \texttt{tabbing} nebo prostředí \texttt{tabular}.

\subsection{Prostředí \texttt{tabbing} }
Při použití \texttt{tabbing} vypadá tabulka následovně:

\begin{tabbing}
vodni melouny \, \= 25,90 \,  \= 2,5kg   \kill
\textbf{Ovoce} \> \textbf{Cena} \> \textbf{Množství} \\
Jablka \> 25,90 \> 3\, kg \\
Hrušky \> 27,40 \> 2,5\, kg \\
Vodní melouny \> 35,-- \> 1\, kus
\end{tabbing}

\noindent Toto prostředí se dá také použít pro sázení algoritmů, ovšem vhodnější je použít 
prostředí \texttt{algorithm} nebo \texttt{algorithm2e} (viz sekce \ref{algoritmy}).


\subsection{Prostředí \texttt{tabular}}
Další možností, jak vytvořit tabulku, je použít prostředí \texttt{tabular}. Tabulky pak 
budou vypadat takto \footnote{Kdyby byl problem s cline, zkuste se podívat třeba sem 
http://www.abclinuxu.cz/tex/poradna/show/325037}:


\begin{table} [h!]
\centering
\catcode`\-=12

\begin{tabular}{|c|c|c|}  \hline
 & \multicolumn{2}{c|}{\textbf{Cena}}\\ \cline{2-3}
\textbf{Měna} & \textbf{Nákup} & \textbf{Prodej} \\ \hline
 EUR & 25,615 & 27,20 \\ 
 GBP & 29,899 & 31,80 \\ 
 USD & 22,571 & 25,51 \\ \hline
\end{tabular}
\caption{Tabulka kurzů k dnešnímu dni}
\end{table}



\begin{table}[h]
\centering

\begin{minipage}{.10\linewidth}
\hspace{2cm}
\centering
\begin{tabular}{|c|c|} \hline
$A$ & $\neg A $ \\ \hline
\textbf{P} & N   \\ \hline
\textbf{O} & O \\ \hline
\textbf{X} & X  \\ \hline
\textbf{N} & P \\\hline
\end{tabular}
\end{minipage}
\begin{minipage}{.20\linewidth}
\centering
\catcode`\-=12
\begin{tabular}{|c|c|c|c|c|c|}\hline
\multicolumn{2}{|c|}{\multirow{2}{*}{$A \wedge B$}} & \multicolumn{4}{|c|}{$B$} \\ \cline{3-6}
\multicolumn{1}{|c}{}&  & P & O & X & N \\ \hline
& P & P & O & X & N \\ \cline{2-6}
\multirow{2}{*}{$A$} & O & O & O & N & N \\ \cline{2-6}
& X & X & N & X & N \\ \cline{2-6}
& N & N & N & N & N \\ \hline

\end{tabular}
\end{minipage}
\hspace{1cm}
\begin{minipage}{.20\linewidth}
\centering
\catcode`\-=12
\begin{tabular}{|c|c|c|c|c|c|}\hline
\multicolumn{2}{|c|}{\multirow{2}{*}{$A \vee B$}} & \multicolumn{4}{|c|}{$B$} \\ \cline{3-6}
\multicolumn{1}{|c}{}&  & P & O & X & N \\ \hline
& P & P & P & P & P \\ \cline{2-6}
\multirow{2}{*}{$A$} & O & P & O & P & O \\ \cline{2-6}
& X & P & P & X & X \\ \cline{2-6}
& N & P & O & X & N \\ \hline

\end{tabular}
\end{minipage}
\hspace{1cm}
\begin{minipage}{.20\linewidth}
\centering
\catcode`\-=12
\begin{tabular}{|c|c|c|c|c|c|}\hline
\multicolumn{2}{|c|}{\multirow{2}{*}{$A \rightarrow B$}} & \multicolumn{4}{|c|}{$B$} \\ \cline{3-6}
\multicolumn{1}{|c}{}&  & P & O & X & N \\ \hline
& P & P & O & X & N \\ \cline{2-6}
\multirow{2}{*}{$A$} & O & P & O & P & O \\ \cline{2-6}
& X & P & P & X & X \\ \cline{2-6}
& N & P & P & P & P \\ \hline

\end{tabular}
\end{minipage}
\caption{Protože Kleeneho trojhodnotová logika už je \uv{zastaralá}, uvádíme si zde příklad čtyřhodnotové logiky}
\end{table}

\pagebreak
\section{Algoritmy} 
\label{algoritmy}

Pokud budeme chtít vysázet algoritmus, můžeme použít prostředí \texttt{algorithm} \footnote{Pro nápovědu, jak zacházet s prostředím algorithm, můžeme zkusit tuhle stránku: \par http://ftp.cstug.cz/pub/tex/CTAN/macros/latex/contrib/algorithms/algorithms.pdf.} nebo \texttt{algorithm2e} \footnote{Pro algorithm2e zase tuhle: http://ftp.cstug.cz/pub/tex/CTAN/macros/latex/contrib/algorithm2e/algorithm2e.pdf.}

\SetKwInput{KwInput}{Input}                % \SetKwInOut{Input}{Input}
\SetKwInput{KwOutput}{Output}              % \SetKwInOut{Output}{Output}

\IncMargin{2em}
\begin{algorithm}
\caption{FastSLAM}


\DontPrintSemicolon

\Indm
\KwInput{$(X_{t-1}, u_t, z_t)$}
\KwOutput{$X_t$}
\Indp
\SetAlgoNoLine

$\overline{X_t} = X_t = 0$

\For{$k = 1$ \KwTo $M$ }{
    $x_t^{[k]} = sample\_motion\_model(u_t, x_{t-1}^{[k]})$\\
    $\omega_t^{[k]} = measurement\_model(z_t, x_{t}^{[k]}, m_{t-1})$\\
    $m_t^{[k]} = updated\_occupancy\_grid(z_t, x_{t}^{[k]}, m_{t-1}^{[k]})$\\
    $\overline{X_t} = \overline{X_t} + \langle x_{x}^{[m]}, \omega_{t}^{[m]} \rangle $\\
    
}
\For{$k = 1$ \KwTo $M$ }{
    draw $i$ with probability $\approx \omega_{t}^{[i]}$\\
    add $\langle x_{x}^{[k]}, m_{t}^{[k]} \rangle$ to $X_t$
    
}

\Return{$X_t$}
\end{algorithm}
\DecMargin{2em}


\section{Obrázky}
Do našich článků můžeme samozřejmě vkládat obrázky. Pokud je obrázkem fotografie,
můžeme klidně použít bitmapový soubor. Pokud by to ale mělo být nějaké schéma nebo
něco podobného, je dobrým zvykem takovýto obrázek vytvořit vektorově.

\begin{figure}[h!]
\begin{center}
\scalebox{0.4}{\includegraphics{etiopan.eps}
\reflectbox{\includegraphics{etiopan.eps}}}
\end{center}
\caption{Malý Etiopánek a jeho bratříček}
\end{figure}

\pagebreak

Rozdíl mezi vektorovým \dots
\begin{figure}[h!]
\begin{center}
\scalebox{0.4}{\includegraphics{oniisan.eps}}
\end{center}
\caption{Vektoroý obrázek}
\end{figure}

\dots a bitmapovým obrázkem \\
\begin{figure}[h!]
\begin{center}
\scalebox{0.6}{\includegraphics{oniisan2.eps}}
\end{center}
\caption{Bitmapový obrázek}
\end{figure}


\noindent se projeví například při zvětšení. 
\par Odkazy (nejen ty) na obrázky 1, 2 a 3, na  
tabulky 1 a 2 a také na algoritmus 1 jsou udělány pomocí 
křížových odkazů. Pak je ovšem potřeba zdrojový soubor přeložit dvakrát.

Vektorové obrázky lze vytvořit i přímo v \LaTeX u, například pomocí prostředí 
\texttt{picture}.

\pagebreak
\begin{figure}[h]
\begin{picture}(350,200)
\thicklines
\put(0,0){\framebox(350,200){}}
\put(15,20){\line(1,0){320} }

\thinlines
\put(310, 180){\circle{30}}
\put(40,20){\line(0,1){100} }
\put(310,20){\line(0,1){100} }

\put(40,120){\line(-1,0){20} }
\put(310,120){\line(1,0){20} }

\put(20,120){\line(2,1){155} }
\put(330,120){\line(-2,1){155} }

\put(20,120){\line(1,0){310} }
\put(40,117){\line(1,0){270} }

\put(50,20){\framebox(20,35){}}
\put(55,35){\framebox(10,15){}}


\put(80,40){\framebox(20,35){}}
\put(80,43){\line(1,0){20} }
\put(80,72){\line(1,0){20} }
\put(83,40){\line(0,1){35} }
\put(97,40){\line(0,1){35} }

\put(105,40){\framebox(20,35){}}
\put(105,43){\line(1,0){20} }
\put(105,72){\line(1,0){20} }
\put(108,40){\line(0,1){35} }
\put(121,40){\line(0,1){35} }


\put(180,20){\framebox(120,80){}}



\end{picture}
\caption{Rodinný dům}
\end{figure}



\end{document}